\chapter{Methodologie}
\label{ch:methodologie}

Het begrip unikernel vraagt om een uitgebreide theoretische kennis van besturingssystemen en virtualisatietechnologieën.

Om deze concepten goed te begrijpen werd er eerst een literatuurstudie uitgevoerd.
Een belangrijk beginpunt was de volgende website \cite{unikernel_systems_unikernels_2016}. De voorgenoemde website heeft een lijst van papers over unikernels en verwijzingen naar implementaties van unikernels.
Veel over virtuele machines was te vinden in thesissen van de voorbije jaren van de opleiding toegepaste informatie. De kennis over software containers werd voornamelijk opgedaan tijdens mijn stage bij Wercker. Een resem boeken over software containers, met als centraal onderwerp Docker, gaven meer inzicht in software containers en hun use cases.

De literatuurstudie vormde de basis voor de eerste hoofdstukken over virtuele machines, containers en unikernels. De bachelorproef focust niet alleen op de mogelijkheden en moeilijkheden van unikernels. De gevolgenen voor de architectuur van applicaties worden nader bekeken.

Het hoofdstuk na unikernels maakt een vergelijking tussen de implementaties van unikernels op het vlak van eenvoud, inzetbaarheid en behandelt de mogelijkheden van elke implementatie. Hiervoor werd er opzoekwerk verricht om de gegevens van de implementaties te vinden.