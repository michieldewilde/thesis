\section{Rumprun}

Bij de rumprun unikernel zal gebruik gemaakt worden van Nginx webserver om een statische website weer te geven. 
Allereerst moeten we de build tools van Rumprun installeren.
Dit wordt gedaan als volgt.

\noindent We halen de Rumprun repository op en alle dependencies
\begin{lstlisting}[language=bash]
  $ git clone http://repo.rumpkernel.org/rumprun
  $ cd rumprun
  $ git submodule update --init
\end{lstlisting}

Verder moeten we toolchain maken voor de omgeving dat zal gebruikt worden. In dit geval is het Xen.
\begin{lstlisting}[language=bash]
  $ ./build-rr.sh xen
\end{lstlisting}

Dit maakt een folder aan genaamd rumprun met daarin een bin folder dit moet toegevoegd worden aan het PATH.
\noindent Toevoegen Rumprun toolchain aan PATH:
\begin{lstlisting}[language=bash]
  $ export PATH=${PATH}:$(pwd)/rumprun/bin
\end{lstlisting}

De toolchain voor Rumprun unikernels te maken is geïnstalleerd en nu moet er een nginx Rumprun unikernel worden opgehaald.
\noindent Ophalen Nginx unikernel:
\begin{lstlisting}[language=bash]
  $ git clone http://repo.rumpkernel.org/rumprun-packages
  $ cd rumprun-packages/
  $ cd nginx/
\end{lstlisting}

Er moet ook een package worden geïnstalleerd om een ISO bestand te maken. Dit zal gebruikt worden voor bestanden te laden in de unikernel.
\noindent Installeren genisoimage:
\begin{lstlisting}[language=bash]
  $ sudo apt-get install genisoimage
\end{lstlisting}

De volgende stap is om Nginx te compileren.
\noindent Compileren Nginx:
\begin{lstlisting}[language=bash]
  $ make
\end{lstlisting}

De unikernel werkt nu al maar moet omgezet worden naar een unikernel die werkt op de Xen hypervisor.
\noindent unikernel voor Xen maken:
\begin{lstlisting}[language=bash]
  $ rumprun-bake xen_pv ./nginx.bin bin/nginx
\end{lstlisting}

Hierbij is nginx.bin de unikernel die moet worden gestart.

\noindent Uitvoeren van de Nginx unikernel:
\begin{lstlisting}[language=bash]
  $ rumprun -T tmp xen -M 64 -i \
  $ -b images/data.iso,/data \ # data van www folder laden (inhoud van statische website)
  $ -I mynet,xenif,bridge=br0 -W mynet,inet,dhcp \ # dhcp instellen
  $ -- nginx.bin -c /data/conf/nginx.conf # laden van unikernel en configuratie
\end{lstlisting}

Het IP van de Nginx webserver is te vinden in de output in het gedeelte van DHCP. Het is het IP adres.

De bridge br0 ,die in MirageOS en Rumprun unikernel wordt gebruikt, is een host-only bridge tussen de host en de virtuele machine om netwerkverkeer tussen beide te laten werken.