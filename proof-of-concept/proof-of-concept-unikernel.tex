\section{Opstelling Unikernel}

Deze opstelling wordt uitgevoerd op een virtuele machine want er worden extensieve veranderingen uitgevoerd op het besturingssysteem. Voor deze opstelling lokaal op te stellen zijn er drie vereisten nodig. Ten eerste de Virtualbox hypervisor voor een virtuele machine op te laten werken. Verder Vagrant om de ontwikkelingomgeving op te stellen. Vagrant en Virtualbox gaan hand in hand om een simpele start te hebben wanneer een bepaalde ontwikkelingomgeving nodig hebt. Vagrant werkt door middel van een bestand (VagrantFile) met alle configuratie erin beschreven. Het haalt eerst een box op waarop de configuratie wordt toegepast. Een box kan een Ubuntu besturingssysteem dat nog niet is aangepast of in deze situatie een besturingssysteem waarop allerhande configuraties zijn uitgevoerd. Packer is de laatste vereiste, het zal gebruikt worden om de box op te stellen die zal gebruikt worden. 

\begin{description}
\item [Virtualbox] - https://www.virtualbox.org/
\item [Vagrant] - https://www.vagrantup.com/
\item [Packer] - https://www.packer.io/ 
\end{description}

Allereest moeten alle voorgenoemde vereisten zijn geïnstalleerd op uw machine voor er verder kan worden gegaan. MirageOS heeft een repository met alle benodigdheden om deze virtuele machine op te stellen. Deze kan gevonden worden op de volgende link: https://github.com/michieldewilde/mirage-vagrant-vms. Haal deze repository lokaal op de machine en navigeer naar de locatie van deze opgehaalde folder. Er is keuze uit ubuntu 14.04, ubuntu 14.10, debian 7.8.0 en xenserver 6.5.0. Er is gekozen voor Ubuntu 14.04 omdat er het minste problemen bij dit besturingssysteem zijn vastgesteld voor deze opstelling. De Makefile bevat voor de gemakkelijkheid commando's voor de virtuele machine op te stellen zonder een lijst van commando's in te geven.

\noindent Het volgende commando zal een nieuwe box aanmaken door het gebruik van Packer:
\begin{lstlisting}[language=bash]
  $ make ubuntu-14.04-box
\end{lstlisting}

\noindent Het volgende commando neemt de box, die is aangemaakt door het vorige commando, en past de configuratie erop toe:
\begin{lstlisting}[language=bash]
  $ make ubuntu-14.04-box
\end{lstlisting}

\noindent Navigeer naar de folder van het gekozen besturingssysteem en ssh in de virtuele machine: 
\begin{lstlisting}[language=bash]
  $ cd ubuntu-14.04 && vagrant ssh
\end{lstlisting}

\section{MirageOS}

MirageOS is gestart vanaf nul met een schone lei. Dit betekende dat veel van de bestaande tools die nu worden gebruikt worden, zoals webservers (Apache, Nginx) en Databases (MySQL), herschreven zouden moeten worden. Het duurt een tijd voordat deze tervoorschijn komen en dit is één van de grootste redenen dat er gekozen is voor een statische applicatie.

De opstelling die gemaakt is in de vorige sectie heeft de toolset van MirageOS ook geïnstalleerd. Verder moet er worden gekeken of de juiste versie van Ocaml en OPAM (package manager) is geïnstalleerd:

\noindent OPAM versie:
\begin{lstlisting}[language=bash]
  $ opam --version
  # De versie moet minstens 1.2.2 zijn.
  1.2.2
\end{lstlisting}

\noindent Ocaml versie:
\begin{lstlisting}[language=bash]
  $ ocaml -version
  # Deze moet 4.02.3 of hoger zijn.
  $ opam switch 4.02.3
\end{lstlisting}

In de login shell moet de omgeving van opam geëvalueerd wanneer er ingelogd wordt. Dit kan gedaan worden door de volgenden lijn toe te voegen aan het ~/.bashrc bestand:

\begin{lstlisting}[language=bash]
  $ eval `opam config env`
\end{lstlisting}

\noindent Verder moeten er ook gekeken of de versie van mirage niet moet upgedate moet worden:
\begin{lstlisting}[language=bash]
  $ opam install mirage
  $ mirage --help
  # Versienummer moet hoger zijn 2.9.0
\end{lstlisting}

Het mirage commando kan gebruikt worden voor applicaties te compileren en uit te rollen.

Als statische website zullen we de website van MirageOS zelf gebruiken. De repository van de website bevindt zich op de volgende link: https://github.com/mirage/mirage-www/. Eerst moet de repository in de virtuele machine worden gehaald. 

\noindent Dit doen we door het git commando te gebruiken:
\begin{lstlisting}[language=bash]
  $ git clone https://github.com/mirage/mirage-www/
\end{lstlisting}

Vooraleer we de omgeving van de applicatie gaan configureren moet het mogelijk worden gemaakt om de Xen hypervisor te laten communiceren met de virtuele machine. Deze twee zijn afgeschermd van elkaar en daarom moet er een TUN/TAP constructie worden gemaakt. 

sudo modprobe tun
\noindent Dit doen we als volgt:
\begin{lstlisting}[language=bash]
  $ sudo apt-get install tunctl
  $ sudo modprobe tun # laden van de tuntap kernel module
  $ sudo tunctl # maak een tap0 interface
  $ sudo ifconfig tap0 10.0.0.1 up # start tap0 op en wijs tap0 naar IP
\end{lstlisting}

Navigeer naar de gehaalde folder van de MirageOS website. 
\noindent Daarna moet de omgeving voor de applicatie worden ingesteld:
\begin{lstlisting}[language=bash]
  $ cd mirage-www
  $ make prepare
  $ cd src
  $ mirage configure --xen \ # configureren voor de Xen hypervisor
      -vv --net direct \ #  direct MirageOS network stack
      --dhcp true \ maak gebruik van DHCP
      --tls false --network=0
\end{lstlisting}

Hierdoor is de omgeving van de applicatie goed ingesteld.
\noindent Het compilen van de unikernel:
\begin{lstlisting}[language=bash]
  $ make
\end{lstlisting}

Eerst moeten nog file blocks worden aanmaken om de afbeeldingen en de stylesheets te kunnen gebruiken:
\noindent Aanmaken file blocks
\begin{lstlisting}[language=bash]
  $ ./make-fat_block1-image.sh
\end{lstlisting}

Verder moet er nog een paar wijzigingen worden aangebracht in het www.xl bestand. Dit bestand wordt doorgegeven naar de Xen launcher om de unikernel te starten.
\noindent Het disk gedeelte moeten worden aangepast naar het volgende:
\begin{lstlisting}
disk = ['format=raw,
         vdev=xvde,
         access=rw,
         target=/home/vagrant/mirage-www/src/fat_block2.img',
        'format=raw,
         vdev=xvdc,
         access=rw,
         target=/home/vagrant/mirage-www/src/fat_block1.img
      ']
\end{lstlisting}


\noindent Uiteindelijk kunnen we de unikernel starten:
\begin{lstlisting}[language=bash]
  $ sudo xl -v create -c www.xl
\end{lstlisting}

Doorheen de output kan gezien worden op welk IP de statische website staat te luisteren. Als er gegaan wordt naar dat IP om de lokale machine dan kan de MirageOS website worden gezien.
\section{Rumprun}

Bij de rumprun unikernel zal gebruik gemaakt worden van Nginx webserver om een statische website weer te geven. 
Allereerst moeten we de build tools van Rumprun installeren.
Dit wordt gedaan als volgt.

\noindent We halen de Rumprun repository op en alle dependencies
\begin{lstlisting}[language=bash]
  $ git clone http://repo.rumpkernel.org/rumprun
  $ cd rumprun
  $ git submodule update --init
\end{lstlisting}

Verder moeten we toolchain maken voor de omgeving dat zal gebruikt worden. In dit geval is het Xen.
\begin{lstlisting}[language=bash]
  $ ./build-rr.sh xen
\end{lstlisting}

Dit maakt een folder aan genaamd rumprun met daarin een bin folder dit moet toegevoegd worden aan het PATH.
\noindent Toevoegen Rumprun toolchain aan PATH:
\begin{lstlisting}[language=bash]
  $ export PATH=${PATH}:$(pwd)/rumprun/bin
\end{lstlisting}

De toolchain voor Rumprun unikernels te maken is geïnstalleerd en nu moet er een nginx Rumprun unikernel worden opgehaald.
\noindent Ophalen Nginx unikernel:
\begin{lstlisting}[language=bash]
  $ git clone http://repo.rumpkernel.org/rumprun-packages
  $ cd rumprun-packages/
  $ cd nginx/
\end{lstlisting}

Er moet ook een package worden geïnstalleerd om een ISO bestand te maken. Dit zal gebruikt worden voor bestanden te laden in de unikernel.
\noindent Installeren genisoimage:
\begin{lstlisting}[language=bash]
  $ sudo apt-get install genisoimage
\end{lstlisting}

De volgende stap is om Nginx te compileren.
\noindent Compileren Nginx:
\begin{lstlisting}[language=bash]
  $ make
\end{lstlisting}

De unikernel werkt nu al maar moet omgezet worden naar een unikernel die werkt op de Xen hypervisor.
\noindent unikernel voor Xen maken:
\begin{lstlisting}[language=bash]
  $ rumprun-bake xen_pv ./nginx.bin bin/nginx
\end{lstlisting}

Hierbij is nginx.bin de unikernel die moet worden gestart.

\noindent Uitvoeren van de Nginx unikernel:
\begin{lstlisting}[language=bash]
  $ rumprun -T tmp xen -M 64 -i \
  $ -b images/data.iso,/data \ # data van www folder laden (inhoud van statische website)
  $ -I mynet,xenif,bridge=br0 -W mynet,inet,dhcp \ # dhcp instellen
  $ -- nginx.bin -c /data/conf/nginx.conf # laden van unikernel en configuratie
\end{lstlisting}

Het IP van de Nginx webserver is te vinden in de output in het gedeelte van DHCP. Het is het IP adres.

De bridge br0 ,die in MirageOS en Rumprun unikernel wordt gebruikt, is een host-only bridge tussen de host en de virtuele machine om netwerkverkeer tussen beide te laten werken.