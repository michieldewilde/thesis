%%---------- Samenvatting -----------------------------------------------------
%%
%% De samenvatting in de hoofdtaal van het document

\chapter*{Samenvatting}

Het efficiënt gebruiken van de middelen die ter beschikking worden gesteld is een belangrijk gegeven binnen informatica. Doorheen de tijd zijn er innovaties gekomen om dit steeds naar een hoger niveau te tillen. Eerst was er sprake van virtuele machines en dan kwamen software containers ter sprake.

Unikernels is een opkomende manier voor het compileren en uitvoeren van applicaties in een productieomgeving. Unikernels vraagt om een andere kijk op programma's in de productieomgeving en hoe er gewerkt wordt met de architectuur en infrastructuur. Deze bachelorproef focust op de use cases voor unikernels. Er wordt nader gekeken naar de mogelijkheden, moeilijkheden en huidige implementaties. Ook de vergelijking tussen software containers en unikernels wordt gemaakt.

Het resultaat van dit onderzoek is dat unikernels kunnen gebruikt worden in één specifieke use case: netwerkapplicaties. Er kan ook nog overwogen worden om unikernels te gebruiken wanneer veiligheid van uitermate belang is.

De belangrijkste conclusie is dat het ecosysteem rond unikernels niet ver genoeg gevorderd is om het te gebruiken in productie en om productiviteit voor de ontwikkelaars te garanderen. Debuggen van unikernels is een groot probleem en dit zou interessant kunnen zijn om te kijken hoe dit zich verder ontwikkeld.
