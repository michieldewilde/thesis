%%========================================================================
%% LaTeX sjabloon voor stage/projectrapport of bachelorproef
%%  HoGent Bedrijf en Organisatie
%%========================================================================

%%========================================================================
%% Preamble
%%========================================================================

\documentclass[pdftex,a4paper,12pt,twoside]{report}

% XXX: Let op: dit sjabloon is gemaakt om dubbelzijdig af te drukken
% Voor enkelzijdig, verwijder ``twoside'' hierboven.

%%---------- Extra functionaliteit ---------------------------------------

\usepackage[utf8]{inputenc}  % Accenten gebruiken in tekst (vb. é ipv \'e)
\usepackage{amsfonts}        % AMS math packages: extra wiskundige
\usepackage{amsmath}         %   symbolen (o.a. getallen-
\usepackage{amssymb}         %   verzamelingen N, R, Z, Q, etc.)
\usepackage[dutch]{babel}    % Taalinstellingen: woordsplitsingen,
                             %  commando's voor speciale karakters
                             %  ("dutch" voor NL)
\usepackage{eurosym}         % Euro-symbool €
\usepackage{geometry}
\usepackage{graphicx}        % Invoegen van tekeningen
\usepackage[pdftex,bookmarks=true]{hyperref}
                             % PDF krijgt klikbare links & verwijzingen,
                             %  inhoudstafel
\usepackage{listings}        % Broncode mooi opmaken
\usepackage{multirow}        % Tekst over verschillende cellen in tabellen
\usepackage{rotating}        % Tabellen en figuren roteren
\usepackage{natbib}          % Betere bibliografiestijlen
\usepackage{fancyhdr}        % Pagina-opmaak met hoofd- en voettekst

\usepackage[T1]{fontenc}     % Ivm lettertypes
\usepackage{lmodern}
\usepackage{textcomp}

\usepackage{lipsum}          % Voor vultekst (lorem ipsum)

%%---------- Layout ------------------------------------------------------

% hoofdingen, enz.
\pagestyle{fancy}
% enkel hoofdstuktitel in hoofding, geen sectietitel (vermijd overlap)
\renewcommand{\sectionmark}[1]{}

% lijn, wordt gebruikt in titelpagina
\newcommand{\HRule}{\rule{\linewidth}{0.5mm}}

% Leeg blad
\newcommand{\emptypage}{
\newpage
\thispagestyle{empty}
\mbox{}
\newpage
}

% Gebruik een schreefloos lettertype ipv het "oubollig" uitziende
% Computer Modern
\renewcommand{\familydefault}{\sfdefault}

% Commando voor invoegen Java-broncodebestanden (dank aan Niels Corneille)
% Gebruik: \codefragment{source/MijnKlasse.java}{Uitleg bij de code}
\newcommand{\codefragment}[2]{ \lstset{%
  language=java,
  breaklines=true,
  float=th,
  caption={#2},
  basicstyle=\scriptsize,
  frame=single,
  extendedchars=\true
}
\lstinputlisting{#1}}

%%---------- Documenteigenschappen ---------------------------------------
%% Vul dit aan met je eigen info:

% Je eigen naam
\newcommand{\student}{Michiel De Wilde}

% De naam van je lector, begeleider, promotor
\newcommand{\promotor}{Bert Van Vreckem}

% De naam van je co-promotor
\newcommand{\copromotor}{Jan Janssen}

% Indien je bachelorproef in opdracht van een bedrijf of organisatie
% geschreven is, geef je hier de naam.
\newcommand{\instelling}{Hogeschool Gent}

% De titel van het rapport/bachelorproef
\newcommand{\titel}{Unikernels}

% Datum van indienen
\newcommand{\datum}{29 mei 2015}

% Faculteit
\newcommand{\faculteit}{Faculteit Bedrijf en Organisatie}

% Soort rapport
\newcommand{\rapporttype}{Scriptie voorgedragen tot het bekomen van de graad van\\Bachelor in de toegepaste informatica}

% Academiejaar
\newcommand{\academiejaar}{2015-2016}

% Examenperiode
%  - 1e semester = 1e examenperiode
%  - 2e semester = 2e examenperiode
%  - tweede zit = 3e examenperiode
\newcommand{\examenperiode}{2e examenperiode}

%%========================================================================
%% Inhoud document
%%========================================================================

\begin{document}

%%---------- Front matter ------------------------------------------------
%% Het voorblad - Hier moet je in principe niets wijzigen.

\begin{titlepage}
  \newgeometry{top=2cm,bottom=1.5cm,left=1.5cm,right=1.5cm}
  \begin{center}

    \begingroup
    \rmfamily
    \includegraphics[width=2.5cm]{img/HG-beeldmerk-woordmerk}\\[.5cm]
    \faculteit\\[3cm]
    \titel
    \vfill
    \student\\[3.5cm]
    \rapporttype\\[2cm]
    Promotor:\\
    \promotor\\
    Co-promotor:\\
    \copromotor\\[2.5cm]
    Instelling: \instelling\\[.5cm]
    Academiejaar: \academiejaar\\[.5cm]
    \examenperiode
    \endgroup

  \end{center}
  \restoregeometry
\end{titlepage}

% Schutblad

\emptypage


\begin{titlepage}
  \newgeometry{top=5.35cm,bottom=1.5cm,left=1.5cm,right=1.5cm}
  \begin{center}

    \begingroup
    \rmfamily
    \faculteit\\[3cm]
    \titel
    \vfill
    \student\\[3.5cm]
    \rapporttype\\[2cm]
    Promotor:\\
    \promotor\\
    Co-promotor:\\
    \copromotor\\[2.5cm]
    Instelling: \instelling\\[.5cm]
    Academiejaar: \academiejaar\\[.5cm]
    \examenperiode
    \endgroup

  \end{center}
  \restoregeometry
\end{titlepage}


\begin{abstract}
% TODO: De "abstract" of samenvatting is een kernachtige (max 1 blz. voor een
% thesis) synthese van het document. In ons geval beschrijf je kort de
% probleemstelling en de context, de onderzoeksvragen, de aanpak en de
% resultaten.
  \lipsum[1-4]
\end{abstract}

\chapter*{Voorwoord}
\label{ch:voorwoord}

Toen ik begon met programmeren had ik geen idee wat er gebeurde in de achtergrond van de computer. Ik starte met de simpele todolist-applicaties om alles te leren over programmeren. Na een tijd kwam ik een black box tegen: het besturingssysteem. Vooral om servers sneller te laten werken en de techniek erachter te leren kennen begon ik aan een zoektocht. Linux was de startplek bij uitstek. Package managers en file systems waren de eerste concepten die mij met verstomming lieten staan. Toen ik meer en meer naar infrastructuur keek begon ik termen te leren en alle handige tips om ervoor te zorgen dat je server altijd beschikbaar is. Toen een paar jaar geleden Docker voor het eerst echt vaart maakte met containers was ik verbaasd. Ik dacht eerst dat dit nooit zou werken. Na een tijd heb ik wel het licht gezien en gebruikte ik containers meer en meer. Toen er gevraagd werd om een onderwerp voor mijn thesis dacht ik meteen en wat volgens mij de volgende stap is: unikernels.



\tableofcontents

% Als je een lijst van afkortingen of termen wil toevoegen, dan hoort die
% hier thuis. Gebruik bijvoorbeeld de ``glossaries'' package.

%%---------- Kern --------------------------------------------------------

\chapter{Inleiding}
\label{ch:inleiding}

Vroeger was de tijd dat je een computer kon gebruiken beperkt. Vooral bij de eerste computers had men problemen om programma's en concepten uit te werken omdat de computer door meerdere mensen gebruikt werd. Een voorbeeld vanuit een situatie uit die tijd is het ontwikkelen van een programma. De source code van een programma werd manueel ingegeven en als een job in een queue gezet. Pas een paar dagen later kon men de resultaten bekijken. Als je een kleine schrijf- en/of logische denkfout maakte dan kon je opnieuw beginnen. De grootste bijdrage tot de ontwikkelsnelheid van programma's is de lengte van de feedbackcyclus: hoe snel kan je je programma testen laten werken. Dit leidt tot een verlies van tijd en dus geld. \\

Timesharing werd uitgevonden om het verlies van tijd te beperken. Bij timesharing konden de gebruikers inloggen op een console en zo computer gebruiken. Dit was een technische uitdaging. De computer zou van de ene context naar de andere moeten kunnen veranderen. Timesharing en verschillende gebruikers op 1 computer zou de basis vormen voor de moderne besturingsystemen. 1 Van de nadelen van timesharing is de isolatie van twee verschiilende processen. Als er een kans bestaat dat ze elkaar kunnen beinvloeden is dit een heel groot probleem. \\

Een besturingsysteem moet kunnen werken op verschillende soorten hardware. Om al deze hardware te ondersteunen moet er een standaard zijn waarop een besturingssysteem kan gebouwd worden. Dit zou betekenen dat veel van de complexiteit van het besturingssysteem naar de hardware zou verhuizen. Spijtig genoeg gebeurt dit nog altijd niet. Dit is 1 van de voornaamste redenen waarom de grootte van een OS tussen de 200MB en 1GB kan liggen. Een aantal besturingssystemen doen niet de moeite om de vele soorten hardware apparatuur te ondersteunen. Zij tonen de specificaties die een hardware apparaat moet hebben wanneer je het besturingssysteem wil gebruiken. \newpage

Doorheen de tijd werden computers meer krachtig en applicaties konden niet alle middelen van de computer ten volle benutten. Dit zorgde voor de creatie van virtual resources. Deze gingen de fysieke hardware simuleren om zo verschillende applicaties tegelijkertijd te laten werken op dezelfde virtuele machine. Zo worden de middelen van een fysieke machine optimaal gebruikt. De algemene term voor dit concept is virtualisatie. Bij het virtualiseren van een server gaat men een fysieke server opdelen in verschillende kleine en geisoleerde delen. Deze delen kunnen dan gebruikt worden door verschillende gebruikers. De voordelen van het virtualiseren van een server zijn de volgende: financieel voordeel (men kan van 1 taak naar meerdere taken gaan op 1 server), het besparen van energie (minder servers gebruiken want ze worden beter benut), betere beschikbaarheid.\\

Er zijn verschillende vormen van virtualisatie. De meest gebruikte vormen bij servers is de virtualisatie van het besturingssysteem en virtualisatie van de hardware.

De virtuele machine is en toepassing van de virtualisatie van de hardware. Een virtuele machine bevindt zich op een fysieke machine door gebruik te maken van een hypervisor. De nadelen van een virtuele machine als de kleinste eenheid is de schaalbaarheid en de veiligheid. Wanneer je je eigen server opstelde dan moet je een besturingssysteem kiezen. Meestal gaan we kiezen voor een Linux distributie. Daarna moet je de software installeren die ervoor zorgt dat de applicatie kan werken. Daaropvolgend moet je je server beveiligen. Het veranderen van de SSH-poort, zorgen dat de root-user niet te bereiken is. IPtables en firewalls opstellen. Voor dit te vergemakkelijken kunnen we gebruikmaken van een configuratie management tool (Chef, Puppet, Ansible...). \\

Een hypervisor is een stuk software, firmware of hardware waarop een VM kan draaien. Een hypervisor zal zich bevinden op een host machine of op "baremetal". De host machine zorgt voor de middelen zoals CPU, RAM, ... Elke virtuele machine die zich bevindt op de host machine zal dan gebruik maken van deze middelen. We spreken van een hosted virtualization hypervisor of een bare-metal hypervisor. De eerste zal zich bevinden op een het besturingssysteem van de host machine en heeft geen directe toegang tot de hardware. Dit heeft als voordeel dat de hardware niet zo belangrijk is maar zorgt voor een extra laag tussen de hardware en de hypervisor. Een goede regel is: "hoe minder lagen we hebben, hoe beter de performantie." Een alternatief is een bare-metal hypervisor. Hierbij is er geen extra laag tussen de hardware en hypervisor. We hebben geen host operating system nodig omdat de hypervisor zich rechtstreeks bevindt op de interfaces van hardware. Dit zorgt voor betere performantie, schaalbaarheid en stabiliteit. Een hypervisor zorgt ervoor dat de virtuele machines het guest OS kan beheren en uitvoeren. \\

Meest bekende voorbeelden van bare-metal hypervisors zijn VMware ESXi, Microsoft Hyper-V en Xen. \\ 

Naast hardware virtualisatie kunnen we ook gebruikmaken van de virtualisatie van het besturingssysteem. Bij deze toepassing van virtualisatie gaan we de mogelijkheden van de kernel van het besturingssysteem gebruiken. De kernel van bepaalde besturingssystemen laat ons toe om meerdere geisoleerde user spaces tegelijkertijd te laten werken. Dit zorgt ervoor dat de dat er maar 1 besturingssysteem en een kernel zijn. De verschillende user spaces maken gebruik van CPU, geheugen en netwerk van de host server. Elke user space heeft zijn eigen configuratie omdat de user spaces op zichzelf staan en geisoleerd zijn de andere user spaces. Tegenover hardware virtualisatie zal de besturingssysteem virtualisatie minder gebruik maken van het geheugen en de CPU omdat er maar 1 kernel is en niet meerdere besturingssystemen. Het opstarten neemt maar een fractie van de tijd in beslag tegenover virtuele machines. Deze user spaces worden ook wel containers genoemd. \\

Meest bekende voorbeelden zijn Docker en Linux Containers. \\

Er werd al aangehaald bij de virtuele machines dat veiligheid een probleem kan zijn. Dit probleem verdwijnt niet bij containers omdat ze 1 kernel delen. Wanneer de beveiliging van 1 container kan geschonden worden dan wordt het direct gemakkelijker om toegang te verschaffen tot de andere containers. Het gebruiken van 1 OS als de basis voor een aantal containers kan ervoor zorgen dat wanneer dit besturingssysteem niet meer opstaat al deze applicaties niet meer werken. \\


\section{Probleemstelling en Onderzoeksvragen}
\label{sec:onderzoeksvragen}

Unikernels zijn een een nieuwe stroom binnen het landschap van besturingssystemen. We hebben al aangehaald dat containers de nieuwe werkwijze is wanneer men applicaties wilt ontwikkelen en schaalbaar wilt maken. Unikernels gaat nog een stap verder. De systeembeheerders moeten dus mee met containers en de veranderingen ondergaan. De vraag is welke veranderingen er zich zullen voordoen wanneer unikernels op de markt komen. Zullen de competenties van de systeembeheerder veranderen? Wordt het opzetten van applicaties meer en meer eenvoudiger of juist niet? We kunnen wel spreken over de opvolger van containers maar is deze al werkbaar in de toekomst? Wat is de impact op beveiliging, meer bepaald aspecten als beschikbaarheid, autorisatie, integriteit en vertrouwelijkheid van gegevens?


\chapter{Methodologie}
\label{ch:methodologie}

% TODO: Hoe ben je te werk gegaan? Verdeel je onderzoek in grote fasen, en
% licht in elke fase toe welke stappen je gevolgd hebt. Verantwoord waarom je
% op deze manier te werk gegaan bent. Je moet kunnen aantonen dat je de best
% mogelijke manier toegepast hebt om een antwoord te vinden op de
% onderzoeksvraag.

\chapter{Corpus}
\label{ch:corpus}

%% TODO: de structuur en titel van deze hoofdstukken hangen af van je
% eigen onderzoek. Elke fase in je onderzoek kan een eigen hoofdstuk krijgen. Kies telkens een gepaste titel. ``Corpus'' is *GEEN* gepaste titel

\chapter{Conclusie}
\label{ch:conclusie}

% TODO: Trek een duidelijke conclusie, in de vorm van een antwoord op de
% onderzoeksvra(a)g(en). Reflecteer kritisch over het resultaat. Zijn er
% zaken die nog niet duidelijk zijn? Heeft het ondezoek geleid tot nieuwe
% vragen die uitnodigen tot verder onderzoek?


\bibliographystyle{apa}
\bibliography{tin-bachproef}

%%---------- Back matter -------------------------------------------------

\listoffigures
\listoftables

\end{document}
