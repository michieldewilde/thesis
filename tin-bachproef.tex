%%========================================================================
%% LaTeX sjabloon voor stage/projectrapport of bachelorproef
%%  HoGent Bedrijf en Organisatie
%%========================================================================

%%========================================================================
%% Preamble
%%========================================================================

\documentclass[pdftex,a4paper,12pt,twoside]{report}

% XXX: Let op: dit sjabloon is gemaakt om dubbelzijdig af te drukken
% Voor enkelzijdig, verwijder ``twoside'' hierboven.

%%---------- Extra functionaliteit ---------------------------------------

\usepackage[utf8]{inputenc}  % Accenten gebruiken in tekst (vb. é ipv \'e)
\usepackage{amsfonts}        % AMS math packages: extra wiskundige
\usepackage{amsmath}         %   symbolen (o.a. getallen-
\usepackage{amssymb}         %   verzamelingen N, R, Z, Q, etc.)
\usepackage[dutch]{babel}    % Taalinstellingen: woordsplitsingen,
                             %  commando's voor speciale karakters
                             %  ("dutch" voor NL)
\usepackage{eurosym}         % Euro-symbool €
\usepackage{geometry}
\usepackage{graphicx}        % Invoegen van tekeningen
\usepackage[pdftex,bookmarks=true]{hyperref}
                             % PDF krijgt klikbare links & verwijzingen,
                             %  inhoudstafel
\usepackage{listings}        % Broncode mooi opmaken
\usepackage{multirow}        % Tekst over verschillende cellen in tabellen
\usepackage{rotating}        % Tabellen en figuren roteren
\usepackage{natbib}          % Betere bibliografiestijlen
\usepackage{fancyhdr}        % Pagina-opmaak met hoofd- en voettekst

\usepackage[T1]{fontenc}     % Ivm lettertypes
\usepackage{lmodern}
\usepackage{textcomp}

\usepackage{lipsum}          % Voor vultekst (lorem ipsum)

%%---------- Layout ------------------------------------------------------

% hoofdingen, enz.
\pagestyle{fancy}
% enkel hoofdstuktitel in hoofding, geen sectietitel (vermijd overlap)
\renewcommand{\sectionmark}[1]{}

% lijn, wordt gebruikt in titelpagina
\newcommand{\HRule}{\rule{\linewidth}{0.5mm}}

% Leeg blad
\newcommand{\emptypage}{
\newpage
\thispagestyle{empty}
\mbox{}
\newpage
}

% Gebruik een schreefloos lettertype ipv het "oubollig" uitziende
% Computer Modern
\renewcommand{\familydefault}{\sfdefault}

% Commando voor invoegen Java-broncodebestanden (dank aan Niels Corneille)
% Gebruik: \codefragment{source/MijnKlasse.java}{Uitleg bij de code}
\newcommand{\codefragment}[2]{ \lstset{%
  language=java,
  breaklines=true,
  float=th,
  caption={#2},
  basicstyle=\scriptsize,
  frame=single,
  extendedchars=\true
}
\lstinputlisting{#1}}

%%---------- Documenteigenschappen ---------------------------------------
%% Vul dit aan met je eigen info:

% Je eigen naam
\newcommand{\student}{Michiel De Wilde}

% De naam van je lector, begeleider, promotor
\newcommand{\promotor}{Bert Van Vreckem}

% De naam van je co-promotor
\newcommand{\copromotor}{Jan Janssen}

% Indien je bachelorproef in opdracht van een bedrijf of organisatie
% geschreven is, geef je hier de naam.
\newcommand{\instelling}{Hogeschool Gent}

% De titel van het rapport/bachelorproef
\newcommand{\titel}{Unikernels}

% Datum van indienen
\newcommand{\datum}{29 mei 2015}

% Faculteit
\newcommand{\faculteit}{Faculteit Bedrijf en Organisatie}

% Soort rapport
\newcommand{\rapporttype}{Scriptie voorgedragen tot het bekomen van de graad van\\Bachelor in de toegepaste informatica}

% Academiejaar
\newcommand{\academiejaar}{2015-2016}

% Examenperiode
%  - 1e semester = 1e examenperiode
%  - 2e semester = 2e examenperiode
%  - tweede zit = 3e examenperiode
\newcommand{\examenperiode}{2e examenperiode}

%%========================================================================
%% Inhoud document
%%========================================================================

\begin{document}

%%---------- Front matter ------------------------------------------------
%% Het voorblad - Hier moet je in principe niets wijzigen.

\begin{titlepage}
  \newgeometry{top=2cm,bottom=1.5cm,left=1.5cm,right=1.5cm}
  \begin{center}

    \begingroup
    \rmfamily
    \includegraphics[width=2.5cm]{img/HG-beeldmerk-woordmerk}\\[.5cm]
    \faculteit\\[3cm]
    \titel
    \vfill
    \student\\[3.5cm]
    \rapporttype\\[2cm]
    Promotor:\\
    \promotor\\
    Co-promotor:\\
    \copromotor\\[2.5cm]
    Instelling: \instelling\\[.5cm]
    Academiejaar: \academiejaar\\[.5cm]
    \examenperiode
    \endgroup

  \end{center}
  \restoregeometry
\end{titlepage}

% Schutblad

\emptypage


\begin{titlepage}
  \newgeometry{top=5.35cm,bottom=1.5cm,left=1.5cm,right=1.5cm}
  \begin{center}

    \begingroup
    \rmfamily
    \faculteit\\[3cm]
    \titel
    \vfill
    \student\\[3.5cm]
    \rapporttype\\[2cm]
    Promotor:\\
    \promotor\\
    Co-promotor:\\
    \copromotor\\[2.5cm]
    Instelling: \instelling\\[.5cm]
    Academiejaar: \academiejaar\\[.5cm]
    \examenperiode
    \endgroup

  \end{center}
  \restoregeometry
\end{titlepage}


\begin{abstract}
% TODO: De "abstract" of samenvatting is een kernachtige (max 1 blz. voor een
% thesis) synthese van het document. In ons geval beschrijf je kort de
% probleemstelling en de context, de onderzoeksvragen, de aanpak en de
% resultaten.
  \lipsum[1-4]
\end{abstract}

\chapter*{Voorwoord}
\label{ch:voorwoord}

% TODO: Vergeet ook niet te bedanken wie je geholpen/gesteund/... heeft
\lipsum[5-6]

\tableofcontents

% Als je een lijst van afkortingen of termen wil toevoegen, dan hoort die
% hier thuis. Gebruik bijvoorbeeld de ``glossaries'' package.

%%---------- Kern --------------------------------------------------------

\chapter{Inleiding}
\label{ch:inleiding}

Vroeger was de tijd die je een computer kon gebruiken beperkt. Vooral bij de eerste computers had men problemen om goeie programma's en concepten uit te werken omdat de machine door meerdere mensen moest gebruikt worden. Een goed voorbeeld daarbij is het ontwikkelen van een programma. De source code van een programma werd manueel ingegeven en als een job in een queue gezet. Pas een paar dagen later kon men de resultaten bekijken. Als je een kleine schrijf- of een logische denkfout maakte dan kan je opnieuw beginnen. De grootste bijdrage tot de ontwikkelsnelheid van programmeurs is de lengte van de feedbackcycle: hoe snel kan je je programma testen en de eventuele fouten eruithalen. Dit leide natuurlijk tot een groot verlies van tijd en geld. Daarom werd timesharing uitgevonden waarbij de gebruikers konden inloggen op een console en zo computer konden gebruiken. Dit zorgde voor een serieuze technische uitdaging. De computer moest van de ene context naar de andere kunnen veranderen. Dit alles vormde de basis voor een operating system(OS). Een OS moet kunnen werken op verschillende hardware en dit zorgt voor een probleem. Om al deze hardware te ondersteunen moet er een standaard zijn waarop een OS kan bouwen. Dit zou betekenen dat veel van de complexiteit van de software laag (OS) naar de hardware zou verhuizen en er dus minder code zou moeten geschreven worden in de software laag. Spijtig genoeg gebeurt dit nog altijd niet. Dit is 1 van de voornaamste redenen waarom de grootte van een OS tussen de 200MB en 1GB kan liggen. Een aantal operating systems doen niet eens de moeite om de vele hardware apparaten de ondersteunen. Zij tonen de specificaties die een hardware apparaat moet hebben wanneer je het OS wil gebruiken. Bij timesharing kunnen we in de problemen komen wanneer twee processen die tegelijkertijd aan het werken zijn niet geisoleerd zijn. Dit kan ervoor zorgen dat ze elkaar beinvloeden wanneer dit niet wenselijk is.

Doorheen de tijd werden computers meer krachtig en applicaties konden niet alle middelen van de computer ten volle benutten. Dit zorgde voor de creatie van virtual resources. Deze gingen de fysieke hardware simuleren om zo verschillende applicaties tegelijkertijd te laten werken op dezelfde virtuele machine. Zo worden de middelen van een fysieke machine optimaal gebruikt. De algemene term is virtualisatie.

Dus we willen applicaties die werken in productie, van elkaar geisoleerd zijn en hun middelen ten volle benutten. De meest bekende voorbeelden zijn de virtuele machine en de container.

De VM is een emulatie van een computer systeem. De nadelen van een VM als de kleinste eenheid is de schaalbaarheid en de veiligheid. Wanneer je je eigen server opstelde dan moest je een OS kiezen. Meestal gaan we kiezen voor een linux distributie wanneer men spreekt over webapplicaties. Daarna moet je de software installeren die ervoor zorgt dat de applicatie kan werken. Daaropvolgend moet je je server beveiligen. Het veranderen van de SSH-poort, zorgen dat de root-user niet te bereiken is. IPtables en firewalls opstellen. Voor dit te vergemakkelijken kunnen we gebruikmaken van een configuratie management tool (Chef, Puppet, Ansible...). Een VM draait op een fysieke machine door gebruik te maken van een hypervisor. 

Een hypervisor is een stuk software, firmware of hardware waarop een VM kan draaien. Een hypervisor zal dan weer draaien op een host machine of op "baremetal". De host machine zorgt voor de middelen zoals CPU, RAM, ... Elke VM die draait op de host machine zal dan gebruik maken van deze middelen. We spreken van een hosted virtualization hypervisor of een bare-metal hypervisor. De eerste zal draaien op een het OS van de host machine en heeft geen directe toegang tot de hardware. Dit heeft als voordeel dat de hardware niet zo belangrijk is maar zorgt wel voor een extra laag tussen de hardware en de hypervisor. Een goede regel is: "hoe minder lagen we hebben, hoe beter de performantie." Een alternatief is een bare-metal hypervisor. Hierbij is er geen extra laag tussen de hardware en hypervisor. We hebben geen host operating system nodig omdat de hypervisor rechtstreeks op de interfaces van hardware draait. Dit zorgt voor betere performantie, schaalbaarheid en stabiliteit. Een hypervisor zorgt ervoor dat de VM's het guest OS kunnen beheren en uitvoeren.

\section{Probleemstelling en Onderzoeksvragen}
\label{sec:onderzoeksvragen}

% TODO: Wees zo concreet mogelijk bij het formuleren van je
% onderzoeksvra(a)g(en). Een onderzoeksvraag is trouwens iets waar nog
% niemand op dit moment een antwoord heeft (voor zover je kan nagaan).
\lipsum[7-20]

\chapter{Methodologie}
\label{ch:methodologie}

% TODO: Hoe ben je te werk gegaan? Verdeel je onderzoek in grote fasen, en
% licht in elke fase toe welke stappen je gevolgd hebt. Verantwoord waarom je
% op deze manier te werk gegaan bent. Je moet kunnen aantonen dat je de best
% mogelijke manier toegepast hebt om een antwoord te vinden op de
% onderzoeksvraag.
\lipsum[21-25]

\chapter{Corpus}
\label{ch:corpus}

%% TODO: de structuur en titel van deze hoofdstukken hangen af van je
% eigen onderzoek. Elke fase in je onderzoek kan een eigen hoofdstuk krijgen. Kies telkens een gepaste titel. ``Corpus'' is *GEEN* gepaste titel
\lipsum[26-75]

\chapter{Conclusie}
\label{ch:conclusie}

% TODO: Trek een duidelijke conclusie, in de vorm van een antwoord op de
% onderzoeksvra(a)g(en). Reflecteer kritisch over het resultaat. Zijn er
% zaken die nog niet duidelijk zijn? Heeft het ondezoek geleid tot nieuwe
% vragen die uitnodigen tot verder onderzoek?
\lipsum[76-80]


\bibliographystyle{apa}
\bibliography{tin-bachproef}

%%---------- Back matter -------------------------------------------------

\listoffigures
\listoftables

\end{document}
